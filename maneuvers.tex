\chapter{Maneuvers}
\banner
\chaptquote{xkcd}{
	The six words you \emph{never} say at NASA: “And besides --
	it works in Kerbal Space Program.”
}

The this chapter, we will see how to change from a circular orbit
to another.

\section{Pro-/retro- grade burn}

The vis-viva equation~(\ref{eql:visviva}) tells us:
\[
\speed{v}^2
=
\mathcal G \mass{M} \left(
	\frac 2 {\dist{r}}
	-
	\frac 2 {\dist{r_a} + \dist{r_p}}
\right)
\]

For example, if we are at an apsis (apo- or peri-) and want to rise the
opposite point, we need to speed up (burn prograde), and slow down to
decrease it; the formula above tells us how much so.

\begin{figure}[H]
	\centering
	\begin{tikzpicture}
		\def\peri{1}
		\def\apoa{1}
		\def\apob{4}
		\node[point=O] (O) at (0,0) {};
		\node (A0) at (0,\apoa) {};
		\node (A1) at (0,\apob) {};
		\orbit[color=blue]{O}{\apoa}{\peri}{0}{360}
		\orbit[color=red] {O}{\apob}{\peri}{0}{360}
		\node[loint=B] (B) at (-\peri,0) {};
	\end{tikzpicture}
	\caption{
		A satellite on the \textcolor{blue}{blue} orbit can
		switch to the \textcolor{red}{red} one by burning
		prograde (speeding up) at $\posit{B}$; conversely it
		can switch from the \textcolor{red}{red} orbit to the
		\textcolor{blue}{blue} one by burning retro grade at
		this same point.
	}
\end{figure}

When searching for good trajectories, we are interested in saving
propellant. According to~(\ref{eql:thrust}), this is the same as saving
for $\speed{\Delta v}$ (althgouh proportionally). If $\dist{r}$ is
the apsis where the burn is performed, $\dist{r_0}$ the opposite apsis
before the burn and $\dist{r_1}$ after:
\begin{align*}
\speed{\Delta v}
&=
|\speed{v_1} - \speed{v_0}|
\\
\frac 1 {\sqrt{\mathcal G \mass{M}}} \speed{\Delta v}
&=
\left|
\sqrt{
	\frac 2 {\dist{r}}
	-
	\frac 2 {\dist{r} + \dist{r_1}}
}
-
\sqrt{
	\frac 2 {\dist{r}}
	-
	\frac 2 {\dist{r} + \dist{r_0}}
}
\right|
\eqtag{eql:burn}
\end{align*}

\section{Hohmann transfer}

Now, assume we are in a circular orbit of radius $\dist{r_0}$ and want to
do a simple transfer to a circular orbit of radius $\dist{r_1}$. During
a Hohmann transfer, we first raise our apoapsis to $\dist{r_1}$ and then
the periapsis (from the new apopsis).

\begin{figure}[H]
	\centering
	\begin{tikzpicture}
		\def\rada{1}
		\def\radb{3}
		\node[point=O] (O) at (0,0) {};
		\orbit[color=blue] {O}{\rada}{\rada}{  0}{360}
		\orbit[color=green]{O}{\radb}{\rada}{180}{360}
		\orbit[color=red]  {O}{\radb}{\radb}{  0}{360}
		\node[loint=B1] (B1) at (-\rada,0) {};
		\node[roint=B2] (B2) at ( \radb,0) {};
	\end{tikzpicture}
	\caption{
		We first switch from the \textcolor{blue}{blue} orbit
		to the \textcolor{green}{green} one by burning at
		$\posit{B1}$ and then from the \textcolor{green}{green}
		one to the \textcolor{red}{red} one by burning at
		$\posit{B2}$.
	}
\end{figure}

\[
\frac 1 {\sqrt{\mathcal G \mass{M}}} \speed{\Delta v}
=
\left|
\sqrt{
	\frac 2 {\dist{r_0}}
	-
	\frac 2 {\dist{r_0} + \dist{r_1}}
}
-
\frac 1 {\sqrt{\dist{r_0}}}
\right|
+
\left|
\frac 1 {\sqrt{\dist{r_1}}}
-
\sqrt{
	\frac 2 {\dist{r_1}}
	-
	\frac 2 {\dist{r_0} + \dist{r_1}}
}
\right|
\]

We can multiply both hands by $\sqrt{\dist{r_0}}$ and set $x = \frac
{\dist{r_1}} {\dist{r_0}}$ to get a simpler expression:

\[
\underbrace{
	\sqrt{\frac {\dist{r_0}} {\mathcal G \mass{M}}}
}_{\alpha}
\speed{\Delta v}
=
\left|
\sqrt{2 - \frac 2 {1 + x}}
-
1
\right|
+
\left|
\frac 1 {\sqrt{x}}
-
\sqrt{\frac 2 x - \frac 2 {1 + x}}
\right|
\]

\begin{figure}[H]
	\centering
	\begin{tikzpicture}
	\begin{axis}[
		samples=\samples,
		domain=0.25:4,
		xtick={0.25,1,2,3,4},
		ytick={0,0.1,0.2,0.3,0.4,0.5,0.6,0.7,0.8,0.9},
		no markers,
		axis lines=left,
		xlabel=$\frac {\dist{r_1}} {\dist{r_0}}$,
		ylabel=$\alpha \speed{\Delta v}$
	]
	\addplot+ {abs(sqrt(2-2/(1+x)) - 1) + abs(1/sqrt(x) - sqrt(2/x - 2/(1+x)))};
	\end{axis}
	\end{tikzpicture}
	\caption{
		Note that decreasing an orbit by half costs about as
		much as the converse (doubling it), but dividing it by
		four costs twice as much as the converse.
	}
\end{figure}

\section{Bi-elliptical transfer}

The idea is to use three burns instead of two.

\begin{figure}[H]
	\centering
	\begin{tikzpicture}
		\def\rada{1}
		\def\radb{3}
		\def\radi{5}
		\node[point=O] (O) at (0,0) {};
		\orbit[color=blue]  {O}{\rada}{\rada}{  0}{360}
		\orbit[color=green] {O}{\radi}{\rada}{180}{360}
		\orbit[color=violet]{O}{\radi}{\radb}{  0}{180}
		\orbit[color=red]   {O}{\radb}{\radb}{  0}{360}
		\node[loint=B1] (B1) at (-\rada,0) {};
		\node[roint=B2] (B2) at ( \radi,0) {};
		\node[roint=B3] (B3) at (-\radb,0) {};
	\end{tikzpicture}
	\caption{
		During a bi-elliptical transfer, we use two
		intermediate orbits (\textcolor{green}{green}, then
		\textcolor{violet}{violet}); the idea is that it will be
		easier to raise the \textcolor{green}{green} periapsis
		from a higher apoapsis
	}
\end{figure}

\begin{align*}
\frac 1 {\sqrt{\mathcal G \mass{M}}} \speed{\Delta v}
=&
\left|
\sqrt{
	\frac 2 {\dist{r_0}}
	-
	\frac 2 {\dist{r_0} + \dist{r_1}}
}
-
\frac 1 {\sqrt{\dist{r_0}}}
\right|
\\
&+
\left|
\sqrt{
	\frac 2 {\dist{r_1}}
	-
	\frac 2 {\dist{r_2} + \dist{r_1}}
}
-
\sqrt{
	\frac 2 {\dist{r_1}}
	-
	\frac 2 {\dist{r_0} + \dist{r_1}}
}
\right|
\\
&+
\left|
\frac 1 {\sqrt{\dist{r_2}}}
-
\sqrt{
	\frac 2 {\dist{r_2}}
	-
	\frac 2 {\dist{r_2} + \dist{r_1}}
}
\right|
\end{align*}

Again, we set $x = \frac {\dist{r_2}} {\dist{r_0}}$ and $y = \frac
{\dist{r_1}} {\dist{r_0}}$ and:
\begin{align*}
\underbrace{
	\sqrt{\frac {\dist{r_0}} {\mathcal G \mass{M}}}
}_{\alpha}
\speed{\Delta v}
=&
\left|
\sqrt{2 - \frac 2 {1 + y}}
-
1
\right|
\\
&+
\left|
\sqrt{\frac 2 y - \frac 2 {x + y}}
-
\sqrt{\frac 2 y - \frac 2 {1 + y}}
\right|
\\
&+
\left|
\frac 1 {\sqrt{x}}
-
\sqrt{\frac 2 x - \frac 2 {x + y}}
\right|
\end{align*}

\begin{figure}[H]
	\centering
	\begin{tikzpicture}
	\begin{axis}[
		samples=\samples,
		domain=5:20,
		no markers,
		axis lines=left,
		xlabel=$\frac {\dist{r_1}} {\dist{r_0}}$,
		ylabel=$\alpha \speed{\Delta v}$,
		legend pos=north west,
	]
	\addplot+ {abs(sqrt(2-2/(1+x)) - 1) + abs(1/sqrt(x) - sqrt(2/x - 2/(1+x)))};
	\addlegendentry{Hohmann}
	\foreach \myy in {10,50}{
		\addplot+ {abs(sqrt(2-2/(1+\myy)) - 1) + abs(sqrt(2/\myy - 2/(x+\myy)) - sqrt(2/\myy - 2/(1+\myy))) + abs(1/sqrt(x) - sqrt(2/x - 2/(x+\myy)))};
		\edef\temp{\noexpand
		\addlegendentry{Bi ($y = \myy$)}
		}\temp
	}
	\end{axis}
	\end{tikzpicture}
\end{figure}

\clearpage
\section{Inclination change}

\begin{important}
Remember, we are only considering \textbf{circular} orbits. The formulas
and derivations below only make sense for circular orbits. We advise
you to set your inclination in a circular orbit before any subsequent
maneuver.
\end{important}

\subsection{(Anti-)normal burn}

Consider the orbital plane in which a satellite is moving. We are
interested in the effect of an acceleration orthogonal to the plane
(normal or antinormal). For this, we study the evolution of the velocity.

\begin{figure}[H]
	\centering
	\begin{tikzpicture}[->,thick]
		\node[boint=P] (P)   at (0,0) {};
		\node (V)   at ( 50:3)    {$\speed{\overrightarrow v}$};
		\node (dV)  at (140:1)    {$\speed{\overrightarrow {\d v}}$};
		\node (VdV) at ($(V)+(dV)$) {$\speed{\overrightarrow {v + \d v}}$};
		\draw[->,thick] (P) -- (V);
		\draw[->,thick] (P) -- (dV);
		\draw[->,thick] (P) -- (VdV);
		\markangle{V}{P}{VdV}{$\d \theta$}{1.5}
	\end{tikzpicture}
	\caption{
		The satellite is heading towards $\speed{\vec v}$ and
		an acceleration is applied to it so that during a time
		$\delay{\dt}$, its velocity is changed by $\speed{\vec{\d
		v}}$.
	}
	\label{fig:normalburn}
\end{figure}

As seen in figure~(\ref{fig:normalburn}), we can easily find the change
in inclination:
\begin{align*}
\angle{\d \theta}
\simeq
\tan \angle{\d \theta}
&=
\frac {\speed{\d v}} {\speed{v}}
\\
\int \frac {\angle{\d \theta}} {\delay{\dt}} \delay{\dt}
&=
\int \frac {\accel{a} \delay{\dt}} {\speed{v}}
\\
\angle{\theta}
&=
\frac {\accel{a} \delay{t}} {\speed{v}}
=
\frac {\speed{\Delta v}} {\speed{v}}
\end{align*}

\subsection{Straight burn}

Doing a $180^{\circ}$ inclination chage using a constant radial burn like
exposed above yields a $\speed{\Delta v}$ proportional to the current
orbital velocity $\speed{v}$: $\speed{\Delta v} = \angle{\pi} \speed{v}
\simeq 3 \speed{v}$. However, simply going retrograde until the speed is
reverse only yields $\speed{\Delta v} = 2 \speed{v}$ for the same result.

You can find another derivation of the cost in \cite[page
2]{incproof,incproof2}.

\begin{figure}[H]
	\centering
	\begin{tikzpicture}[->,thick]
		\node[boint=P] (P)   at (0,0) {};
		\coordinate (V0) at (40:3);
		\coordinate (V1) at (80:3);
		\coordinate (VV) at ($(V1)-(V0)$);
		\draw (P) -- +(V0) node[above]{$\speed{\overrightarrow {v_0}}$};
		\draw (P) -- +(V1) node[above]{$\speed{\overrightarrow {v_1}}$};
		\draw[red] (P) -- +(VV) node[above]{$\speed{\overrightarrow {\Delta v}}$};
		\draw[red] (V0) -- +(VV);
		\markangle{V0}{P}{V1}{$\theta$}{1}
	\end{tikzpicture}
	\caption{
		A rotation of angle $\angle{\theta}$ from velocity vector
		$\speed{\vec{v_0}}$ to $\speed{\vec{v_1}}$ is done in
		a straight change in speed $\speed{\Delta v}$.
	}
\end{figure}

We need to compute $\speed{\Delta v}$ for given $\speed{v} =
\speed{|\vec{v_0}|} = \speed{|\vec{v_1}|}$ and $\angle{\theta}$. Because
the triangle is isosceles, the altitude and the median from $P$ are
one so:
\[
\speed{\Delta v}
=
\left|2 \speed{v} \sin \frac {\angle{\theta}} 2\right|
\]

For $\angle{\theta} = \angle{\pi}$, we get $\speed{\Delta v} = 2
\speed{v}$ which is the expected result.

\begin{figure}[H]
	\centering
	\begin{tikzpicture}
	\begin{axis}[
		samples=\samples,
		domain=-180:180,
		xtick={-180,-135,...,180},
		no markers,
		axis lines=left,
		xlabel=$\theta$ ($^{\circ}$),
		ylabel=$\frac {\speed{\Delta v}} {\speed{v}}$,
	]
		\addplot+ {abs(2*sin(x/2))};
	\end{axis}
	\end{tikzpicture}
	\caption{
		An inclination change of about $\angle{30^{\circ}}$
		already costs half the orbital speed; $\angle{60^{\circ}}$
		costs as much as the orbital speed.
	}
\end{figure}

\subsection{Bi-elliptical inclination change}

Whatever the method used for the inclination change, the cost is
proportional to the current orbital speed. Thus, it is more efficient to
do such a maneuver at low speed (e.g. at apoapsis). /u/ObsessedWithKSP
demonstrated a maneuver similar to the bi-elliptical transfer for a more
efficient plane change \cite{biincchange,biincchange2}.

A formal derivation of the optimal inclination change has been published
by /u/listens\_to\_galaxies \cite{incproof,incproof2}.

\begin{figure}[H]
	\centering
	\begin{tikzpicture}
		\def\rada{2}
		\def\radi{5}
		\def\width{\rada^0.5 * \radi^0.5}
		\node[point=O] (O) at (0,0) {};
		\node[loint=B1] (B1) at (-\rada,0) {};
		\node[roint=B2] (B2) at ( \radi,0) {};
		\begin{scope}[canvas is xy plane at z=0,color=blue]
		\orbit{O}{\rada}{\rada}{0}{360}
		\orbit{O}{\radi}{\rada}{180}{360}
		\draw (-\rada,-\width) rectangle (\radi,\width);
		\end{scope}
		\begin{scope}[canvas is zx plane at y=0,rotate=90,color=red]
		\draw (-\rada,-\width) rectangle (\radi,\width);
		\orbit{O}{\radi}{\rada}{0}{180}
		\orbit{O}{\rada}{\rada}{0}{360}
		\end{scope}
	\end{tikzpicture}
	\caption{
		Starting in the \textcolor{blue}{blue} plane on the
		circular orbit, the spacecraft first burns prograde in
		$\posit{B1}$ to raise its apoapsis to $\posit{B2}$;
		once there, its speed is lower and it can proceed to
		the inclination change to the \textcolor{red}{red}
		plane effectively; finaly, it burns retrograde back in
		$\posit{B1}$ to return to a circular orbit.
	}
\end{figure}

\section{Radial in/out burn}

\section{Arbitrary burn}
