\chapter{Basics}
\banner
\chaptquote{Carl Sagan}{
	Imagination will often carry us to worlds that never were. But
	without it we go nowhere.
}
\csname banner1\endcsname



\section{Units}


\subsection{Dimensions}

There exist different units for each kind of measure (e.g. \dist{distance}
in \dist{meters}, \dist{feet}, etc. To avoid confusion, we agreed on which
units should be prefered; they are called SI Units (for \emph{Système
International}, International System).

\begin{figure}[H]
	\centering
	\begin{tabular}{l|l|l}
		Dimension        & SI unit       & Other units                  \\ \hline
		\angle{angle}    & radian (-)    & turn (-), degree ($^{\circ}$), gradian (gon) \\ \hline
		\delay{duration} & second (s)    & hour (h), day (d), year (y)  \\ \hline
		\dist {distance} & meter (m)     & feet, miles, light-year (ly) \\ \hline
		\speed{speed}    & m/s           & km/h, knot                   \\ \hline
		\mass {mass}     & kilogram (kg) & ton, pound                   \\ \hline
		       pressure  & pascal (Pa)   & atmosphere (atm)             \\ \hline
	\end{tabular}
	\caption{In this context, a dimension is a type of measure.}
\end{figure}

\begin{remark}
A \dist{light} year is the distance that a particle of light can travel
in a \delay{year} of time. For comparison, it takes light a little more
than \delay{eight minutes} (\delay{8min}) to get from the Sun to the
Earth, meaning that the Sun is \dist{8 light-minutes} away from the
Earth. Kerbin is about \dist{45 light-seconds} away from Kerbol.
\end{remark}



\subsection{Prefixes}

We are used to refer to \dist{10,000 meters} (\dist{m}) as \dist{10
kilometers} (\dist{km}). “kilo-” is a prefix meaning “thoussand”;
it is the most used prefix but others exist:

\begin{figure}[H]
	\centering
	\begin{tabular}{l|l|l|l}
		kilo- (k-) & mega- (M-) & giga- (G-) & tera- (T-) \\
		\hline
		$10^3$     & $10^6$     & $10^9$     & $10^{12}$  \\
	\end{tabular}
	\caption{Common prefixes}
\end{figure}

\begin{remark}
There are also prefixes to decrease the value of an unit:

\begin{figure}[H]
	\centering
	\begin{tabular}{l|l|l|l}
		milli- (m-) & micro- ($\mu$-) & nano- (n-) & pico- (p-) \\
		\hline
		$10^{-3}$   & $10^{-6}$       & $10^{-9}$  & $10^{-12}$ \\
	\end{tabular}
	\caption{Smaller prefixes}
\end{figure}
\end{remark}


\subsection{Conversion}

We sometimes need to switch the unit used for a measure. For example,
let us convert $\speed{50 km/h}$ to SI units. We know that $\dist{km} = \dist{1000 m}$ and $\delay{h} =
\delay{3600 s}$ so:
\[
\speed{50 km/h}
= 50 (\dist{1000m})/(\delay{3600s})
= 50 \times 1000 / 3600 \speed{m/s}
= \speed{14 m/s}
\]

This particular example shows how easy it is to include units in
computations. As we will see below, having the units is useful when
considering more complex expressions.


\subsection{Addition (and substraction)}

An addition can only be done with the same kind of measure (e.g. a
distance can only be summed with another distance). Now, consider the
following operation:
\[
\dist{x} = \dist{2 ly} + \dist{4,730.3 Tm}
\]

Knowing that $\dist{ly} = \dist{9.4607 Tm}$:
\begin{align*}
\dist{x}
&= \dist{2 ly} + \dist{4,730.3 Tm} \\
&= 2 \times \dist{9.4607 Tm} + \dist{4,730.3 Tm} \\
&= (18.92146 + 4,730.3) \dist{Tm} \\
&= \dist{23.6518 Tm}
\end{align*}

Conversly, $\dist{Tm} = \dist{1/9.4607 ly}$ and:
\begin{align*}
\dist{x}
&= \dist{2 ly} + \dist{4,730.3 Tm} \\
&= \dist{2 ly} + (4,730.3 / 9.4607) \dist{ly} \\
&= (2 + 0.5) \dist{ly} \\
&= \dist{2.5 ly}
\end{align*}

Of course, the two ways are equivalent and $\dist{2.5 ly} = \dist{23.6518
Tm}$.


\subsection{Multiplication (and division)}

We can create new units pretty easily. For example, if some object travels
$d = 15 m$ in $t = 3 s$, we can define a value $v = d / t = (15m) / (3s)
= 5 m/s$. This particular value is called the mean velocity. Conversely,
assume the object has traveled at $50 km/h$ for $30s$; then, the distance
it has gone through is:
\[
\dist{d}
= \speed{v} \times \delay{t}
= (\speed{50 km/h}) \times (\delay{30s})
= 14 \speed{m/s} \times 30 \delay{s}
= (14 \times 30) (\dist{m}/\strike[red]{\delay{s}} \times \strike[red]{\delay{s}})
= \dist{417 m}
\]



\section{Functions}

\begin{figure}[H]
	\centering
	\begin{tikzpicture}[->]
	\node[point=O] (O) at (0,0) {};
	\node          (E) at (5,0) {x};
	\node[point=C] (C) at (3,0) {};
	\draw (O) -> (E);
	\end{tikzpicture}
	\caption{$\posit{C}$ is moving towards the right at speed $\speed{v}$}
\end{figure}

Consider a car $\posit{C}$ moving along a straight road at a speed of
$\speed{v}$ (e.g. $50km/h$). We define $\dist{x}$ as the distance from
the origin $\posit{O}$ to the car $\posit{C}$, e.g. $\dist{OC}$. After
$\delay{t}$ time has passed (e.g. $\delay{t} = \delay{10s}$), we know
that $\dist{x} = \speed{v} \delay{t}$; in other words, $\dist{x}$ is
\textbf{function} of $\delay{t}$:
\[
\dist{x}(\delay{t}) = \speed{v} \delay{t}
\]

This notation gives us a general formula to compute $\dist{x}$ for any
given value of $\delay{t}$. For example:
\[
\dist{x}(\delay{1h})
= \speed{50km/h} \times \delay{1h}
= \dist{50 km}
\]

As another example, consider a mass point $(\posit{O}, \mass{M})$. Then
the value of the associated gravitation at a point $\posit{P}$ is:
\[
\accel{g}(\dist{r}) = \frac {\mathcal G \mass{M}} {\dist{r}^2}
\]
where $\mathcal G = 6.67 \times 10^{-11} N m^2/kg^2$ is a known value
(gravitational constant) and $\dist{r}$ is the distance from $O$ to $P$.



\section{Derivative}


\subsection{Illustration}

As previously, consider a car $C$ moving away from $\posit{O}$ with
$\dist{x} = \dist{OC}$; this time we do not know how fast it is going, and
it can accelerate or decelerate. First, because the speed is changing,
we need a way to define it at any moment. We know how to define the
\textbf{mean speed}:
\[
\speed{v}_{\mathrm{mean}} = \dist{x}(\delay{t}) / \delay{t}
\]

Like any mean, it depends on the speed since the beginning. For example,
if the car has been going slowly during a long time and them abruptly
accelerate, the mean speed remain low while the actual speed is higher.

Instead of using the mean speed betwen times $\delay{0s}$ and
$\delay{t}$. Instead, we will skip the beginning of the journey to start
at time $\delay{t}$ and end after a short delay $\delay{\dt}$:
\[
\frac {
	\overbrace{\dist{x}(\delay{t}+\delay{\dt})}^{\text{final position}}
	- \overbrace{\dist{x}(\delay{t})}^{\text{skipping before}}
} {\underbrace{\delay{\dt}}_{\text{amount of time traveled}}}
\]

As the value of $\delay{\dt}$ approaches $\delay{0}$, this
“short-mean” approaches the actual speed; it is called the
\textbf{instant speed}. Because the part on the fraction is small too,
we usually note it $\dist{\d x}$. We then end up with:
\[
\frac {\dist{\d x}(\delay{t})} {\delay{\dt}}
\text{ or }
\frac {\dist{\d x}} {\delay{\dt}}(\delay{t})
\text{ or }
\frac {\d} {\delay{\dt}} \dist{x}(\delay{t})
\]

Like we wanted, the instant speed depends on the time: it is a function
as well; it is called the derivative of $\dist{x}$. Because $\dist{x}$
was a distance (e.g. m) and we derived with respect to a delay $\delay{t}$
(e.g. s), the derivative is a speed (e.g. m/s).

\begin{remark}
An even shorter notation than $\frac {\d} {\delay{\dt}}$ (only when we
use $\delay{\dt}$) is to just add a dot: $\speed{\dot x}(\delay{t})$.
\end{remark}


\subsection{Definition}

From what is above, we get that the formal definition for the derivative
of a function $f$ is:
\[
\frac {\d f(x)} {\d x}
= \lim_{\d x \rightarrow 0} \frac {f(x+\d x) - f(x)} {\d x}
\]


\subsection{Second derivative}

The instant speed is a function. Let us derivate it again. A variation in
speed through time is an acceleration.  The derivative of the derivative
is called the second derivative: \[
\frac {\d^2 \dist{x}} {\delay{\dt}^2}
= \frac {\d \speed{\dot x}} {\delay{\dt}}
= \accel{\ddot x}
\]


\subsection{Common derivatives}

\[
\frac {\d} {\d x} (1)   = 0
\]
\[
\frac {\d} {\d x} (x)   = 1
\]
\[
\frac {\d} {\d x} (x^2) = 2x
\]
\[
\frac {\d} {\d x} (x^n) = n x^{n-1}
\]
\[
\frac {\d} {\d x} (a f)
= a \frac {\d} {\d x} f
\]
\[
\frac {\d} {\d x} (f + g)
= \frac {\d} {\d x} f
+ \frac {\d} {\d x} g
\]
\[
\frac {\d} {\d x} (f \times g)
= f \times \frac {\d} {\d x} g
+ g \times \frac {\d} {\d x} f
\]
\[
\frac {\d} {\d x} (f(g(x)))
= \frac {\d} {\d x} g(x)
\times \frac {\d} {\d x} f(g(x))
\]



\section{Integrals}


\subsection{Concept}

Once more, consider the car on the road. We are given its speed
$\speed{\dot x}(\delay{t})$ through the time. Because we know how fast the
car was going at any single instant, we can deduce how much it traveled
from time $\delay{t_1}$ to $\delay{t_2}$ for example. The computation
of the position depending on the speed is
noted with the integral symbol (which just means sum of small elements):
\[
\int_{\delay{t_1}}^{\delay{t_2}} \speed{\dot x}(\delay{t}) \delay{\dt}
= \int_{\delay{t_1}}^{\delay{t_2}} \frac {\dist{\d x}} {\strike[red]{\delay{\dt}}} \strike[red]{\delay{\dt}}
= \sum_{\delay{t_1}}^{\delay{t_2}} \dist{\d x}
= \underbrace{
	\dist{x}(\delay{t_2}) - \dist{x}(\delay{t_1})
}_{[\dist{x}]_{\delay{t_1}}^{\delay{t_2}}}
\]


\subsection{Illustration}

For example, if $\speed{\dot x} = \accel{2m/s^2} \times \delay{t}$, then:
\begin{align*}
\dist{x}(\delay{30s}) - \dist{x}(\delay{0})
&= \int_{\delay{0}}^{\delay{30s}} \accel{2m/s^2} \times \delay{t} \delay{\dt} \\
&= \left(\int_{\delay{0})}^{\delay{30s}} 2 \delay{t} \delay{\dt}\right) \accel{m/s^2} \\
&= [\delay{t}^2]_{\delay{0}}^{\delay{30s}} \accel{m/s^2} \\
&= (900s^2 - 0s^2) \accel{m/s^2} \\
&= \dist{900m}
\end{align*}

In particular, with we are given the additional information that $\posit{C}$
did start at $\posit{O}$, i.e.  $\dist{x}(\delay{0}) = \dist{0}$, then:
\[
\dist{x}(\delay{30s}) = \dist{900m}
\]


\subsection{Antiderivative}

More generally, in the previous example, we could write:
\[
\dist{x}(\delay{t}) - \dist{x}(\delay{0})
= \int_{\delay{0}}^{\delay{30s}} \accel{2m/s^2} \times \delay{t} \delay{\dt}
= \delay{t}^2 \accel{m/s^2}
\]

Another way to write it is:
\[
\dist{x}(\delay{t})
= t^2 \accel{m/s^2} + \dist{C}
\]
where $\dist{C} = \dist{x}(\delay{0})$ is a value independent of
$\delay{t}$ which depends on the \textbf{initial conditions} (e.g. the
position of the car at the initial instant). Each of the possible
expression of $\dist{x}$ (depending on $\dist{C}$) is a primitive of
$\speed{\dot x}$.



\section{Differential equations}


\subsection{Definition}

A differential equation is an equation whose unknown is a function in
which a derivative of this function appears. For example:
\[
\frac {\d f} {\d x} = 3x^2
\]

This is a differential equation and we already know how to solve it
(find the expression of $f$). Here, $f(x) = x^3 + C$ (there are several
possible solutions).


\subsection{Exponential}

The exponential function is defined as $f(x) = e^x$ where $e$ is a
mathematical constant whose value is about $2.718$. It was picked so that:
\[
\frac {\d f} {\d x} = f
\]

In other words:
\[
\frac {\d} {\d x} (e^x) = e^x
\]

\begin{remark}
If we define $f(x) = e^{g(x)}$ instead, derivation rules
gives us:
\[
\frac {\d f(x)} {\d x}
= g'(x) e^{g(x)}
\]
so that $\frac {\d f} {\d x} = g' f$.
\end{remark}


\subsection{First order}

Now, consider the following equation:
\[
\frac {\d f} {\d x} = f
\]

We already know that $f(x) = e^x$ is a solution; however, so is $f(x) =
2 e^x$. Actually, the set of solutions to this equation is the functions
$f(x) = C e^x$ where $C$ is any constant value.

The remark we made before tell us how to solve a differential equation
of the form:
\[
\frac {\d f} {\d x} = h f
\]
where $h$ is also a function. We just need to find $g$ such that $g' =
h$, i.e. the solutions are:

\[
f(x) = C e^{\int_0^x h(x) \d x}
\]


\section{Geometric integrals}

Circle circumference:
\[
\dist{d}
= \oint_C \dist{\d s}
= \int_{\angle{0}}^{\angle{2\pi}} \dist{R} \angle{\d \theta}
= \dist{R} \left[ \theta \right]_{\angle{0}}^{\angle{2\pi}}
= 2\pi \dist{R}
\]

Disk area:
\[
\area{A}
= \iint_D \area{\d A}
= \int_{\dist{r}=\dist{0}}^{\dist{R}}
  \int_{\angle{\theta}=\angle{0}}^{\angle{2\pi}}
  \dist{\d r} \times \dist{r} \angle{\d \theta}
= 2\pi \int_{\dist{0}}^{\dist{R}} \dist{r} \dist{\d r}
= 2\pi \left[\frac 1 2 \dist{r}^2\right]_{\dist{0}}^{\dist{R}}
= \pi \dist{R}^2
\]

Sphere area:
\[
\area{A}
= \oiint_S \area{\d A}
= \int_{\angle{\theta}=\angle{0}}^{\angle{\pi}}
  \int_{\angle{\phi}=\angle{0}}^{\angle{2\pi}}
  (\dist{R} \sin \angle{\phi} \angle{\d \theta})
  (\dist{R} \angle{\d \phi})
= 2\pi \dist{R}^2 \int_{\angle{0}}^{\angle{\pi}} \sin \angle{\phi} \angle{\d \phi}
= 2\pi \dist{R}^2 \left[ - \cos \phi \right]_{\angle{0}}^{\angle{\pi}}
= 4\pi \dist{R}^2
\]

Sphere enclosed volume:
\[
\vol{V}
= \iiint_B \vol{\d V}
= \int_{\dist{r}=\dist{0}}^{\dist{R}}
  \int_{\angle{\theta}=\angle{0}}^{\angle{\pi}}
  \int_{\angle{\phi}=\angle{0}}^{\angle{2\pi}}
  (\dist{r} \sin \angle{\phi} \angle{\d \theta}) (\dist{r} \d \angle{\phi}) \dist{\d r}
= 2 \pi
  \int_{\dist{r}=\dist{0}}^{\dist{R}}
  \int_{\angle{\theta}=\angle{0}}^{\angle{\pi}} \sin \angle{\phi} \angle{\d \phi} \dist{r}^2 \dist{\d r}
= 2 \pi
  \left[\frac 1 3 \dist{r}^3\right]_{\dist{0}}^{\dist{R}}
  \left[- \cos \angle{\phi}\right]_{\angle{0}}^{\angle{\pi}}
= \frac 4 3 \pi \dist{R}^3
\]
