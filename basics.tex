\chapter{Basics}
\banner
\chaptquote{Carl Sagan}{
	Imagination will often carry us to worlds that never were. But
	without it we go nowhere.
}
\csname banner1\endcsname



\section{Units}


\subsection{Dimensions}

There exist different units for each kind of measure (e.g. \dist{distance}
in \dist{meters}, \dist{feet}, etc. To avoid confusion, we agreed on which
units should be prefered; they are called SI Units (for \emph{Système
International}, International System).

\begin{figure}[H]
\centering
\begin{tabular}{l|l|l}
Dimension        & SI unit       & Other units                   \\ \hline
\angle{angle}    & radian (-)    & turn (-), degree ($^{\circ}$) \\ \hline
\delay{duration} & second (s)    & hour (h), day (d), year (y)   \\ \hline
\dist {distance} & meter (m)     & feet, miles, light-year (ly)  \\ \hline
\speed{speed}    & m/s           & km/h, knot                    \\ \hline
\mass {mass}     & kilogram (kg) & ton, pound                    \\ \hline
       pressure  & pascal (Pa)   & atmosphere (atm)              \\ \hline
\end{tabular}
\caption{In this context, a dimension is a type of measure.}
\end{figure}

\begin{remark}
A \dist{light} year is the distance that a particle of light can travel
in a \delay{year} of time. For comparison, it takes light a little more
than \delay{eight minutes} (\delay{8~min}) to get from the Sun to the
Earth, meaning that the Sun is \dist{8~light-minutes} away from the
Earth. Kerbin is about \dist{45~light-seconds} away from Kerbol.
\end{remark}



\subsection{Prefixes}

We are used to refer to \dist{10,000 meters} (\dist{m}) as \dist{10
kilometers} (\dist{km}). “kilo-” is a prefix meaning “thoussand”;
it is the most used prefix but others exist:

\begin{figure}[H]
\centering
\begin{tabular}{l|l|l|l}
kilo- (k-) & mega- (M-) & giga- (G-) & tera- (T-) \\
\hline
$10^3$     & $10^6$     & $10^9$     & $10^{12}$  \\
\end{tabular}
\caption{Common prefixes}
\end{figure}

\begin{remark}
There are also prefixes to decrease the value of an unit:

\begin{figure}[H]
\centering
\begin{tabular}{l|l|l|l}
milli- (m-) & micro- ($\mu$-) & nano- (n-) & pico- (p-) \\
\hline
$10^{-3}$   & $10^{-6}$       & $10^{-9}$  & $10^{-12}$ \\
\end{tabular}
\caption{Smaller prefixes}
\end{figure}
\end{remark}


\subsection{Conversion}

We sometimes need to switch the unit used for a measure. For example,
let us convert $\speed{50~km/h}$ to SI units. We know that $\dist{km}
= \dist{1000~m}$ and $\delay{h} = \delay{3600~s}$ so:

\[
\speed{50~km/h}
= 50 (\dist{1000~m})/(\delay{3600~s})
= 50 \times 1000 / 3600 \speed{m/s}
= \speed{14~m/s}
\]

This particular example shows how easy it is to include units in
computations. As we will see below, having the units is useful when
considering more complex expressions.


\subsection{Addition (and substraction)}

An addition can only be done with the same kind of measure (e.g. a
distance can only be summed with another distance). This is seems obvious,
but simply checking that the values that are being some are actually of
the same kind can help locate errors early and save tremendous amounts
of time. Now, consider the following operation:

\[
\dist{x} = \dist{2~ly} + \dist{4,730.3~Tm}
\]

We do not know how to sum arbitrary units (for instance \dist{m}
with \mass{kg}). However, we do know that that $\dist{ly} =
\dist{9.4607~Tm}$. Thus, we can replace it in the expression:

\begin{align*}
\dist{x}
&= 2 \times \dist{ly} + \dist{4,730.3~Tm} \\
&= 2 \times \dist{9.4607~Tm} + \dist{4,730.3~Tm} \\
&= (18.92146 + 4,730.3) \dist{Tm} \\
&= \dist{23.6518~Tm}
\end{align*}

Conversly, we could also have said that $\dist{Tm} = \dist{1/9.4607~ly}$
and then:

\begin{align*}
\dist{x}
&= \dist{2~ly} + \dist{4,730.3~Tm} \\
&= \dist{2~ly} + (4,730.3 / 9.4607) \dist{ly} \\
&= (2 + 0.5) \dist{ly} \\
&= \dist{2.5~ly}
\end{align*}

Of course, the two ways are equivalent and we can check that
$\dist{2.5~ly} = \dist{23.6518~Tm}$.


\subsection{Multiplication (and division)}

We can create new units pretty easily. For example, if some object
travels a distance $\dist{d} = \dist{15~m}$ in a duration $\delay{t}
= \delay{3~s}$, we can define the velocity $\speed{v} = \dist{d} /
\delay{t} = \dist{15~m} / \delay{3~s} = \speed{5~m/s}$. Conversely,
assume the object has traveled at a velocity $\speed{50~km/h}$ for a
duration $\delay{30~s}$; then, the distance it has gone through is:

\[
\dist{d}
= \speed{v} \times \delay{t}
= (\speed{50~km/h}) \times (\delay{30~s})
= 14~\speed{m/s} \times 30~\delay{s}
= (14 \times 30) (\dist{m}/\strike[red]{\delay{s}} \times \strike[red]{\delay{s}})
= \dist{417~m}
\]

By doing the operations on both the numerical values and the units,
we know what our result is: in this case, it's a distance, and it is
expressed in meters.



\section{Functions}

\begin{figure}[H]
\centering
\begin{tikzpicture}[->]
\node[point=O] (O) at (0,0) {};
\node          (E) at (5,0) {x};
\node[point=C] (C) at (3,0) {};
\draw (O) -> (E);
\end{tikzpicture}
\caption{$\posit{C}$ is moving towards the right at speed $\speed{v}$}
\end{figure}

Let us consider a car $\posit{C}$ moving along a straight road at a
speed of $\speed{v}$ (e.g. $\speed{50~km/h}$). The position of the car,
$\posit{C}$, can be determined by the distance from $\posit{C}$ to an
arbitrary fix point $\posit{O}$ (the \textbf{origin}). We will note this
distance $\dist{x}$ and we have thus $\dist{x} = \dist{OC}$.

Say we are interested in the variations of the position of the car as
time changes and say write the current time $\delay{t}$ as the delay since
the car was at the origin ($\posit{C} = \posit{O}$).

In other words, we are interested in $\dist{x}$ as a \textbf{function}
of $\delay{t}$. After some time $\delay{t}$ has passed (e.g. $\delay{t}
= delay{10~s}$), we know that $\dist{x} = \speed{v} \times \delay{t}$. We
write it:

\[
\dist{x}(\delay{t}) = \speed{v} \times \delay{t}
\]

This notation gives us a general formula to compute $\dist{x}$ for any
given value of $\delay{t}$. For example:

\[
\dist{x}(\delay{1~h})
= \speed{50~km/h} \times \delay{1~h}
= \dist{50~km}
\]

As another example, it is known that the intensity of the light emitted by a star decreases proportionnaly to the square of the distance to the star. This can be written as:

\[
L(\dist{r}) = \frac C {\dist{r}^2}
\]

where $C$ is some \textbf{constant} value (i.e. independent from
$\dist{r}$) which is to be determined experimentally.



\section{Derivatives}


\subsection{Definition}

Assume we know the position $\dist{x}(\delay{t})$ of the car for any
instant $\delay{t}$ and we want to determine the velocity of the car at
a given instant $\delay{t_0}$.

\pgfmathdeclarefunction{examplepos}{1}{\pgfmathparse{e^(2.2*(#1)) - e^(10-5*(#1)) + 22024}}
\begin{figure}[H]
\centering
\begin{tikzpicture}
\begin{axis}[
	samples=\samples,
	domain=0:5,
	ticks=none,
	no markers,
	axis lines=left,
	xlabel=$\delay{t}$,
	ylabel=$\dist{x}$,
	legend pos=north west,
	clip=false,
]
\addplot{examplepos(x)};
\node[roint=A] (A) at (axis cs:4, {examplepos(4)}) {};
\node[boint={\delay{t_0}}] (t) at (A |- {axis cs:0,0}) {};
\node[loint={\dist{x}(\delay{t_0})}] (x) at (A -| {axis cs:0,0}) {};
\draw[dashed] (t) -- (A) -- (x);
\addlegendentry{$\dist{x}(\delay{t})$}
\end{axis}
\end{tikzpicture}
\caption{
	The horizontal axis represents the passage of time, the vertical
	axis the position. The blue line shows the current position of
	the current at every instant. We are interested in the instant
	$\delay{t_0}$; this is noted $\posit{A}$ on the graph.
}
\end{figure}

The velocity is a the \textbf{variation} of position through time. Thus,
to know how fast the car is going at time $\delay{t_0}$, we need to look
at the position of the car at two different instants. We already have
$\delay{t_0}$; let us also consider $\delay{t_0+h}$ for some arbitrary
value $\delay{h}$.

The difference in position between instants $\delay{t_0}$ and
$\delay{t_0+h}$ is thus $\dist{x}(\delay{t_0+h}) - \dist{x}(\delay{t_0})$;
a shorter notation for this is $\dist{\Delta x}(\delay{t_0})$. The
value $\delay{h}$ is not shown because we do not really care about
it. The Greek letter $\Delta$, for $d$, is generally used to denote a
\textbf{d}ifference (here, the difference in position).

Notice that, the bigger $\delay{h}$, the bigger we expect this difference
to be; to compensate for this, we will divide by how much time has passed,
which is to say $\delay{t_0+h} - \delay{t_0} = \delay{\Delta t_0})$:

\[
\frac {\dist{\Delta x}(\delay{t_0})} {\delay{\Delta t_0}}
\]

This value is the \textbf{mean velocity} from instant $\delay{t_0}$ to
instant $\delay{t_0+h}$. However, the mean velocity is a value that
only gives a general idea of the speed on some period of time. In this
duration, the \textbf{instant velocity} (actual speed) can very a lot
and the mean velocity would then be far off to these values.

\begin{figure}[H]
\centering
\begin{tikzpicture}
\begin{axis}[
	samples=\samples,
	domain=0:5,
	ticks=none,
	no markers,
	axis lines=left,
	xlabel=$\delay{t}$,
	ylabel=$\dist{x}$,
	legend pos=north west,
	clip=false,
]
\addplot{examplepos(x)};
\node[roint=A] (A) at (axis cs:4.0, {examplepos(4.0)}) {};
\node[roint=B] (B) at (axis cs:4.9, {examplepos(4.9)}) {};
\node[boint={\delay{t_0}}]   (t0) at (A |- {axis cs:0,0}) {};
\node[boint={\delay{t_0+h}}] (t1) at (B |- {axis cs:0,0}) {};
\node[loint={\dist{x}(\delay{t_0})}]   (x0) at (A -| {axis cs:0,0}) {};
\node[loint={\dist{x}(\delay{t_0+h})}] (x1) at (B -| {axis cs:0,0}) {};
\draw[dashed] (t0) -- (A) -- (x0);
\draw[dashed] (t1) -- (B) -- (x1);
\draw[color=red,shorten <=-2cm,shorten >=-1cm] (A) -- (B);
\addlegendentry{$\dist{x}(\delay{t})$}
\end{axis}
\end{tikzpicture}
\caption{
	We add a point $\posit{B}$ to the previous graph at time
	$\delay{t_0}+h$. The mean velocity from $\posit{A}$ to $\posit{B}$
	can be thought as the slope of the red line $(AB)$.
}
\end{figure}

Since
we expect the speed to not change a lot on short periods of time, a
natural solution is to consider the mean velocity over shorter durations.

\begin{figure}[H]
\centering
\foreach \i in {0,...,3}{
	\begin{subfigure}{0.4\textwidth}
	\begin{tikzpicture}[scale=0.5]
		\begin{axis}[
		samples=\samples,
		domain=0:5,
		ticks=none,
		no markers,
		axis lines=left,
		legend pos=north west,
		clip=false,
		]
	\addplot{examplepos(x)};
	\node[roint=A] (A) at (axis cs:4.0, {examplepos(4.0)}) {};
	\node[loint=B] (B) at (axis cs:{4.0+0.9/(2^\i)}, {examplepos(4.0+0.9/(2^\i))}) {};
	\draw[color=red,shorten <=-2cm,shorten >=-1cm] (A) -- (B);
	\end{axis}
	\end{tikzpicture}
	\end{subfigure}
}
\caption{
	The closer to $\posit{A}$ we pick $\posit{B}$, the best the red
	line matches the curve for velocity at $\posit{A}$.
}
\end{figure}

So, as we pick shorter and shorter durations $\delay{h}$, the value
$\delay{\Delta t_0}$ becomes smaller, but so does $\dist{\Delta
x}$. Often, we will notice that the mean velocity over $\delay{h}$
seems to becomes closer and closer to a particular value. Instead of
continuing to choose smaller and smaller values of $\delay{h}$, we will
pick this values and call it the \textbf{limit} of $\frac {\dist{\Delta
x}(\delay{t_0})} {\delay{\Delta t_0}}$ as $\delay{h}$ tends to $0$
(becomes smaller and smaller):

\[
\lim_{\delay{h} \to \delay{0}} \frac {\dist{\Delta x}(\delay{t_0})} {\delay{\Delta t_0}}
\]

The limit of such an expression is called the \textbf{derivative} of
$\posit{x}$ at $\delay{t_0}$. We have a shorter way to note this:

\[
\frac {\dist{\d x}(\delay{t_0})} {\delay{\dt}}
\]

We can then remark that we have actually defined the derivative for any
$\delay{t_0}$. Thus, we have a new function that let us evaluate the
velocity at any $\delay{t_0}$.

\[
\frac {\dist{\d x}} {\delay{\dt}}
\]

\begin{remark}
An even shorter notation is $\speed{\dot x}$. This notation is only used
for derivatives with respect to time.
\end{remark}


\subsection{Second derivative}

The velocity is the derivative of the position. As a function, it can
itself fluctuate and we can be interested in these variations. The
derivative of the velocity is the \textbf{acceleration}: $\accel{a} =
\accel{\dot v}$.

A shorter way of saying that the acceleration is the derivative of
\textit{the derivative of the position}, is to say that the acceleration
is the second derivative of the position: $\accel{a} = \accel{\ddot x}$.


\subsection{Formal derivation}

While we now have a way to compute the derivative of a function at a
given point, it is not accurate: while we do get a better approximation
by taking a smaller value for $h$, the result is still an approximation
and can sometimes stay far off.

Instead, we can look at the expressions to determine the exact value
for the limit. For instance, let us consider the function $f(x) = 12x$
and let us search for $\frac {\d f(x)} {\d x}$ for any given $x$. First:

\[
\frac {\Delta f(x)} {\Delta x}
= \frac {f(x+h) - f(x)} {(x+h) - x}
= \frac {12(x+h) - 12x} {h}
= \frac {12\strike[red]{h}} {\strike[red]{h}}
= 12
\]

We now look at the value $\frac {\Delta f(x)} {\Delta x}$ as $\Delta x$
gets small; in this case, it happens to always be $12$, and does not
depend on $\Delta x$. Thus, however small $\Delta x$, the value is
$12$, and:

\[
\frac {\d f(x)} {\d x}
= \lim_{h \to 0} \frac {\Delta f(x)} {\Delta x}
= 12
\]

That way, we know the exaxt value of derivative of $f$ in any point. Let
us take a second example with $g(x) = 7x^2$:

\[
\frac {\Delta g(x)} {\Delta x}
= \frac {7(x+h)^2 - 7x^2} {h}
= \frac {7(x^2 + 2xh + h^2) - 7x^2} {h}
= \frac {7x\strike[red]{h} + h^{\strike[red]{2}}} {\strike[red]{h}}
= 7x + h
\]

Here, the expression does depend on $h$. However, the smaller $h$ gets,
the less influence it has on the sum: the value is becoming closer and
closer to $7x$. Thus: $\frac {\d g(x)} {\d x} = 7x$.


\subsection{Common derivatives}

\[
\frac {\d} {\d x} (1)   = 0
\]
\[
\frac {\d} {\d x} (x)   = 1
\]
\[
\frac {\d} {\d x} (x^2) = 2x
\]
\[
\frac {\d} {\d x} (x^n) = n x^{n-1}
\]
\[
\frac {\d} {\d x} (a f)
= a \frac {\d} {\d x} f
\]
\[
\frac {\d} {\d x} (f + g)
= \frac {\d} {\d x} f
+ \frac {\d} {\d x} g
\]
\[
\frac {\d} {\d x} (f \times g)
= f \times \frac {\d} {\d x} g
+ g \times \frac {\d} {\d x} f
\]
\[
\frac {\d} {\d x} (f(g(x)))
= \frac {\d} {\d x} g(x)
\times \frac {\d} {\d x} f(g(x))
\]



\section{Integrals}


\subsection{Concept}

Once more, consider the car on the road. We are given its speed
$\speed{\dot x}(\delay{t})$ through the time. Because we know how fast the
car was going at any single instant, we can deduce how much it traveled
from time $\delay{t_1}$ to $\delay{t_2}$ for example. The computation
of the position depending on the speed is
noted with the integral symbol (which just means sum of small elements):
\[
\int_{\delay{t_1}}^{\delay{t_2}} \speed{\dot x}(\delay{t}) \delay{\dt}
= \int_{\delay{t_1}}^{\delay{t_2}} \frac {\dist{\d x}} {\strike[red]{\delay{\dt}}} \strike[red]{\delay{\dt}}
= \sum_{\delay{t_1}}^{\delay{t_2}} \dist{\d x}
= \underbrace{
	\dist{x}(\delay{t_2}) - \dist{x}(\delay{t_1})
}_{[\dist{x}]_{\delay{t_1}}^{\delay{t_2}}}
\]


\subsection{Illustration}

For example, if $\speed{\dot x} = \accel{2~m/s^2} \times \delay{t}$, then:
\begin{align*}
\dist{x}(\delay{30~s}) - \dist{x}(\delay{0~s})
&= \int_{\delay{0}}^{\delay{30~s}} \accel{2~m/s^2} \times \delay{t} \delay{\dt} \\
&= \left(\int_{\delay{0})}^{\delay{30~s}} 2 \delay{t} \delay{\dt}\right) \accel{m/s^2} \\
&= [\delay{t}^2]_{\delay{0}}^{\delay{30~s}} \accel{m/s^2} \\
&= (900~s^2 - 0~s^2) \accel{m/s^2} \\
&= \dist{900~m}
\end{align*}

In particular, with we are given the additional information that $\posit{C}$
did start at $\posit{O}$, i.e.  $\dist{x}(\delay{0}) = \dist{0}$, then:
\[
\dist{x}(\delay{30~s}) = \dist{900~m}
\]


\subsection{Antiderivative}

More generally, in the previous example, we could write:
\[
\dist{x}(\delay{t}) - \dist{x}(\delay{0})
= \int_{\delay{0}}^{\delay{30~s}} \accel{2~m/s^2} \times \delay{t} \delay{\dt}
= \delay{t}^2 \accel{m/s^2}
\]

Another way to write it is:
\[
\dist{x}(\delay{t})
= t^2 \accel{m/s^2} + \dist{C}
\]
where $\dist{C} = \dist{x}(\delay{0})$ is a value independent of
$\delay{t}$ which depends on the \textbf{initial conditions} (e.g. the
position of the car at the initial instant). Each of the possible
expression of $\dist{x}$ (depending on $\dist{C}$) is a primitive of
$\speed{\dot x}$.



\section{Differential equations}


\subsection{Definition}

A differential equation is an equation whose unknown is a function in
which a derivative of this function appears. For example:
\[
\frac {\d f} {\d x} = 3x^2
\]

This is a differential equation and we already know how to solve it
(find the expression of $f$). Here, $f(x) = x^3 + C$ (there are several
possible solutions).


\subsection{Exponential}

The exponential function is defined as $f(x) = e^x$ where $e$ is a
mathematical constant whose value is about $2.718$. It was picked so that:
\[
\frac {\d f} {\d x} = f
\]

In other words:
\[
\frac {\d} {\d x} (e^x) = e^x
\]

\begin{remark}
If we define $f(x) = e^{g(x)}$ instead, derivation rules
gives us:
\[
\frac {\d f(x)} {\d x}
= g'(x) e^{g(x)}
\]
so that $\frac {\d f} {\d x} = g' f$.
\end{remark}


\subsection{First order}

Now, consider the following equation:
\[
\frac {\d f} {\d x} = f
\]

We already know that $f(x) = e^x$ is a solution; however, so is $f(x) =
2 e^x$. Actually, the set of solutions to this equation is the functions
$f(x) = C e^x$ where $C$ is any constant value.

The remark we made before tell us how to solve a differential equation
of the form:
\[
\frac {\d f} {\d x} = h f
\]
where $h$ is also a function. We just need to find $g$ such that $g' =
h$, i.e. the solutions are:

\[
f(x) = C e^{\int_0^x h(x) \d x}
\]


\section{Geometric integrals}

Circle circumference:
\[
\dist{d}
= \oint_C \dist{\d s}
= \int_{\angle{0}}^{\angle{2\pi}} \dist{R} \angle{\d \theta}
= \dist{R} \left[ \theta \right]_{\angle{0}}^{\angle{2\pi}}
= 2\pi \dist{R}
\]

Disk area:
\[
\area{A}
= \iint_D \area{\d A}
= \int_{\dist{r}=\dist{0}}^{\dist{R}}
  \int_{\angle{\theta}=\angle{0}}^{\angle{2\pi}}
  \dist{\d r} \times \dist{r} \angle{\d \theta}
= 2\pi \int_{\dist{0}}^{\dist{R}} \dist{r} \dist{\d r}
= 2\pi \left[\frac 1 2 \dist{r}^2\right]_{\dist{0}}^{\dist{R}}
= \pi \dist{R}^2
\]

Sphere area:
\[
\area{A}
= \oiint_S \area{\d A}
= \int_{\angle{\theta}=\angle{0}}^{\angle{\pi}}
  \int_{\angle{\phi}=\angle{0}}^{\angle{2\pi}}
  (\dist{R} \sin \angle{\phi} \angle{\d \theta})
  (\dist{R} \angle{\d \phi})
= 2\pi \dist{R}^2 \int_{\angle{0}}^{\angle{\pi}} \sin \angle{\phi} \angle{\d \phi}
= 2\pi \dist{R}^2 \left[ - \cos \phi \right]_{\angle{0}}^{\angle{\pi}}
= 4\pi \dist{R}^2
\]

Sphere enclosed volume:
\[
\vol{V}
= \iiint_B \vol{\d V}
= \int_{\dist{r}=\dist{0}}^{\dist{R}}
  \int_{\angle{\theta}=\angle{0}}^{\angle{\pi}}
  \int_{\angle{\phi}=\angle{0}}^{\angle{2\pi}}
  (\dist{r} \sin \angle{\phi} \angle{\d \theta}) (\dist{r} \d \angle{\phi}) \dist{\d r}
= 2 \pi
  \int_{\dist{r}=\dist{0}}^{\dist{R}}
  \int_{\angle{\theta}=\angle{0}}^{\angle{\pi}} \sin \angle{\phi} \angle{\d \phi} \dist{r}^2 \dist{\d r}
= 2 \pi
  \left[\frac 1 3 \dist{r}^3\right]_{\dist{0}}^{\dist{R}}
  \left[- \cos \angle{\phi}\right]_{\angle{0}}^{\angle{\pi}}
= \frac 4 3 \pi \dist{R}^3
\]
