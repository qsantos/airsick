\chapter{Mechanics}
\banner
\chaptquote{Isaac Newton}{
	Every body continues in its state of rest, or of uniform motion
	in a right line, unless it is compelled to change that state by
	forces impressed upon it.
}



\section{Referential}

A referential is the object you use as a landmark (origin) to keep track
of interesting points. A system of coordinates if the kind of data you
use to store the position of these points relatively to the origin.


\subsection{Cartesian coordinates}

Cartesian coordinates are the most common. You basically choose two
directions on give the how much you have to go on each coordinate to
get from the origin to the point.

\begin{figure}[H]
	\centering
	\begin{tikzpicture}
		\def \axlen {3}
		\def \x {1}
		\def \y {2}
		\node[boint=O]  (O) at (0,0) {};
		\node[point=P]  (P)  at (\x,\y) {};
		\node[boint=\x] (Px) at (\x,0) {};
		\node[loint=\y] (Py) at (0,\y) {};
		\draw[thick,->] (O) -- (\axlen,0) node[anchor=west] {$x$};
		\draw[thick,->] (O) -- (0,\axlen) node[anchor=south]{$y$};
		\draw (P) edge[dashed] (Px);
		\draw (P) edge[dashed] (Py);
	\end{tikzpicture}
	\caption{$\posit{P}$ is at coordinates $\posit{(1,2)}$ in this referential}
\end{figure}

Cartesian coordinates can be used in three dimensions by adding one axis
(usually noted $z$).


\subsection{Polar coordinates}

Polar coordinates work in two dimensions. On value is simply the distance
to the origin while a second is the angle of $\posit{P}$ with a fixed
porientation.

\begin{figure}[H]
	\centering
	\begin{tikzpicture}
		\def \R  {3}
		\def \th {40}
		\node[point=O] (O) at (0,0) {};
		\node          (X) at (0:\R) {};
		\node[point=P] (P) at (\th:\R) {};
		\draw (O) -- (X);
		\draw (O) -- node[above left]{$\dist{r}$} (P);
		\markangle{X}{O}{P}{$\theta$}{\R/3};
	\end{tikzpicture}
	\caption{The polar coordinates of $\posit{P}$ are $\posit{(\theta:r)}$}
\end{figure}


\subsection{Polar coordinates}

We define the polar base as $(\hat r, \hat \theta) = ((\cos
\angle{\theta}, \sin \angle{\theta}), (-\sin \angle{\theta}, \cos
\angle{\theta}))$. It means that the point $(\angle{\theta}:\dist{r})$
has Cartesian coordinates $r(\cos \angle{\theta}, \sin
\angle{\theta})$. First, we remark that:
\[
\left\{
\begin{aligned}
\frac {\d} {\dt} \hat r      &= \dot \theta (- \sin \angle{\theta},   \cos \angle{\theta}) = \dot \theta \hat \theta \\
\frac {\d} {\dt} \hat \theta &= \dot \theta (- \cos \angle{\theta}, - \sin \angle{\theta}) = - \dot \theta \hat r \\
\end{aligned}
\right.
\]

Then, when we derive a vector in polar base, we get:
\[
\speed{\dot {\vec r}}
= \frac {\d} {\delay{\dt}} \Big(\dist{r} \hat r\Big)
= \speed{\dot r} \hat r + \dist{r} \dot \theta \hat \theta
\]

By deriving again, we get the acceleration:
\[
\accel{\ddot {\vec r}}
= \frac {\d} {\delay{\dt}} \Big(\speed{\dot r} \hat r + \dist{r} \dot \theta \hat \theta\Big)
= (\accel{\ddot r} - \dist{r} {\dot \theta}^2) \hat r
  + (2 \speed{\dot r} \dot \theta + \dist{r} \ddot \theta) \hat \theta
\]


\subsection{Spherical coordinates}

Spherical coordinates are simply the generalization of polar
coordinates to three dimensions. To the usual coordinates $\posit{r}$
and $\angle{\theta}$, we add a third one, $\angle{\phi}$, which is the
angle with the polar plane:

\begin{figure}[H]
	\centering
	\tdplotsetmaincoords{60}{110}
	\begin{tikzpicture}[tdplot_main_coords]
		\def \r  {4}
		\def \th {30}
		\def \ph {60}

		\node[loint=O]  (O) at (0,0,0) {};
		\draw[thick,->] (O) -- (4,0,0) node[anchor=north east]{$x$};
		\draw[thick,->] (O) -- (0,4,0) node[anchor=north west]{$y$};
		\draw[thick,->] (O) -- (0,0,4) node[anchor=south]{$z$};

		\tdplotsetcoord{P}{\r}{\th}{\ph}
	\draw[red,-stealth] (O) -- (P) node[above right] {$P$};
	\draw[red,dashed]   (O) -- (Pxy);
	\draw[red,dashed]   (P) -- (Pxy);
	\tdplotsetthetaplanecoords{\ph}
	\tdplotdrawarc                       {(O)}{1}{0}{\ph}{anchor=north}     {$\phi$}
	\tdplotdrawarc[tdplot_rotated_coords]{(O)}{2.5}{0}{\th}{anchor=south west}{$\theta$}
	\end{tikzpicture}
	\caption{The polar coordinates of $\posit{P}$ are $\posit{(\theta:r)}$}
\end{figure}



\section{Newtonian mechanics}


\subsection{Center of mass}

For the sake of simplicity, we will pretend that objects are simple points
(e.g. $\posit{P}$) associated to their mass (e.g. $\mass{m}$); such points
(e.g. $(\posit{P},\mass{m})$) are called mass-points. For a given object,
the point we will use is called the center of mass (CoM). It is easy to
find for regular and uniform objects (e.g. a sphere).

We will usually refer to the position $\posit{P}$ by using the vector
$\dist{\vec r} = \dist{\overrightarrow{OP}}$. The velocity is simply
the derivative of the position: $\speed{\vec v} = \speed{\dot{\vec
r}}$; the acceleration is the derivative of the velocity: $\accel{a} =
\accel{\dot{\vec v}} = \accel{\ddot{\vec r}}$.


\subsection{Newton's second law}

If we consider a point-mass $(\posit{P}, \mass{m})$ which is subjected
to forces $\force{\vec F}$. Note that $\force{\vec F}$ represent the
sum of all the forces exerted on $\posit{P}$.
\[
\accel{\vec a} = \frac 1 {\mass{m}} \force{\vec F}
\]

\begin{remark}
When $\force{\vec F} = 0$, the acceleration is null as well and the
speed is constant (in practise, there are forces of friction, which
slows objects down). This is Newton's first law (chapter quote).
\end{remark}



\section{Shell theorem}

We consider a sphere of center $\posit{C}$, radius $\dist{R}$ and uniform
density $\mu$ whose center is at distance $\dist{r}$ of mass point
$(\mass{m},\posit{P})$. We wish to infer the force $\vec g$ exerted
by $\posit{C}$ on $\posit{P}$. We will use spherical coordinates and
center the referential on $\posit{P}$ since it is the one point not
moving when integrating.

\begin{figure}[H]
	\centering
	\begin{tikzpicture}
	\def \r  {5}
	\def \R  {2}
	\def \th {40}
	\draw (0,0) circle (\R);
	\node[point={C,\mu}]         (C) at (0, 0)   {};
	\node[point={P,\mass{m}}]    (P) at (\r,0)   {};
	\node[point={Q,\mass{\d m}}] (Q) at (\th:\R) {};
	\draw (C) -- node[below]{$\dist{r}$} (P);
	\draw (P) -- node[above]{$\dist{s}$} (Q);
	\draw (Q) -- node[above]{}  (C);

	\markangle{C}{P}{Q}{$\psi$}{1}
	\markangle{Q}{C}{P}{$\theta$}{1}

	\end{tikzpicture}
\end{figure}

Because of the symmetry around the axis $(PC)$, we already know that
$\vec g$ will be in the same direction as $\dist{\overrightarrow{PC}}$.
\begin{align*}
\accel{g}
&= \iiint_S \d \accel{\vec g} \cdot \frac {\dist{\overrightarrow{PC}}} {\dist{PC}} \\
%
&= \iiint \frac {\mathcal G \mu \vol{\d V}} {\dist{s}^2} \cos \angle{\psi} \\
%
&= \mathcal G \mu
   \int_{\dist{\rho}=\dist{\rho_-}}^{\dist{\rho^+}}
   \int_{\angle{\psi}=\angle{0}}^{\angle{\alpha}}
   \int_{\angle{\theta}=\angle{0}}^{\angle{2\pi}}
   \frac {\cos \angle{\phi}} {\dist{\rho}^2}
   \times (\dist{\d \rho})
   (\dist{\rho} \sin \angle{\psi} \angle{\d \theta}) (\dist{\rho} \angle{\d \psi}) \\
%
&= 2\pi \mathcal G \mu
   \int_{\dist{\rho}=\dist{\rho_-}}^{\dist{\rho^+}}
   \int_{\angle{\psi}=\angle{0}}^{\angle{\alpha}}
   \cos \angle{\psi} \sin \angle{\psi} \angle{\d \psi} \dist{\d \rho} \\
%
\eqtag{eql:before}
&= 2\pi \mathcal G \mu
   \int_{\angle{\psi}=\angle{0}}^{\angle{\alpha}}
   2 \sqrt{\dist{R}^2 - \dist{r}^2 \sin^2 \angle{\psi}}
   \cos \angle{\psi} \sin \angle{\psi} \angle{\d \psi} \dist{\d \rho} \\
%
\eqtag{eql:after}
&= 4\pi \mathcal G \mu
   \int_{\dist{u}=\strike[green]{\dist{R}}}^{\strike[green]{\dist{0}}}
   \dist{u}
   \sqrt{\strike[blue]{1} - \frac {\strike[blue]{\dist{R}^2-\dist{u}^2}} {\dist{r}^2}}
   \frac {\strike[red]{\sqrt{\dist{R}^2-\dist{u}^2}}} {\dist{r}}
   \frac {\strike[green]{-}\dist{u}} {
	 \strike[blue]{\sqrt{\dist{r}^2 - \dist{R}^2 + \dist{u}^2}}
	\strike[red]{\sqrt{\dist{R}^2 - \dist{u}^2}}
   } \dist{\d u} \\
%
&= 4\pi \mathcal G \mu
   \frac 1 {\dist{r}^2}
   \int_{\dist{0}}^{\dist{R}}
   \dist{u}^2 \dist{\d u} \\
%
&= \mathcal G \underbrace{\frac 4 3 \pi \dist{R}^3}_{= \mass{M}} \times \frac 1 {\dist{r}^2}
\end{align*}

To get from~(\ref{eql:before}) to~(\ref{eql:after}), we substitute
$\dist{u} = \sqrt{\dist{R}^2 - \dist{r}^2\sin^2\angle{\psi}}$
for $\angle{\psi}$. This means that $\angle{\psi} = \arcsin \frac
{\sqrt{\dist{r}^2 - \dist{u}^2}} {\dist{r}}$ and subsequently:
\[
\angle{\d \psi}
= \frac 1 {\sqrt{1 - \frac {\dist{R}^2-\dist{u}^2} {\dist{r}^2}}}
  \times \frac {-2 \dist{u} \dist{\d u}} {2 \dist{r} \sqrt{\dist{R}^2 - \dist{u}^2}} \\
= - \frac {\dist{u}} {
	\sqrt{\dist{r}^2
	- \dist{R}^2
	+ \dist{u}^2} \sqrt{\dist{R}^2
	- \dist{u}^2}
} \dist{\d u}
\]

We then use the relations $\sin(\arcsin x) = x$ and $\cos(\arcsin x) =
\sqrt{1 - x^2}$ for $0 \leq x \leq \frac {\pi} 2$.



\section{Sphere of incluence}

Consider spherical bodies $(\posit{P_1}, \mass{M_1})$ and $(\posit{P_2},
\mass{M_2})$ and a point-mass $(\posit{P}, \mass{m})$ between those
two ($\posit{P} \in [\posit{P_1}, \posit{P_2}]$). We want to estimate
how much $\posit{P_1}$ amounts in the gravitation forces exerted on
$\posit{P}$. According to the previous section we don't need to know
the radius of the bodies and can just pretend they are point-masses as
well. The intensity of forces $\force{F_1}$ and $\force{F_2}$ respectively
exerted by $\posit{P_1}$ and $\posit{P_2}$ are:

\[
\force{F_1} = \mathcal G \frac {\mass{M_1} \mass{m}} {\dist{P_1 P}^2}
\text{ and }
\force{F_2} = \mathcal G \frac {\mass{M_2} \mass{m}} {\dist{P_2 P}^2}
\]

Thus, with $\dist{r} = \dist{P_1 P}$ and $\dist{P_2 P} = \dist{P_1 P_2}
- \dist{r}$:

\[
\frac {\force{F_1}} {\force{F_2}}
= \frac {\mass{M_1} (\dist{P_1 P_2} - \dist{r})^2} {\mass{M_2} \dist{r}^2}
= \frac {\mass{M_1}} {\mass{M_2}} \left(\frac {\dist{P_1 P_2}} {\dist{r}} - 1\right)^2
\]

So, if we want $F_1$ to be at least $p$ of the exerted force, we want:

\begin{align*}
\frac {\force{F_1}} {\force{F_1} + \force{F_2}} \ge p
& \Leftrightarrow 1 + \frac {\force{F_2}} {\force{F_1}} \le p^{-1} \\
& \Leftrightarrow \frac {\force{F_1}} {\force{F_2}} \ge \frac 1 {p^{-1} - 1} \\
& \Leftrightarrow \frac {\mass{M_1}} {\mass{M_2}} \left(\frac {\dist{P_1 P_2}} {\dist{r}} - 1\right)^2 \ge \frac 1 {p^{-1} - 1} \\
& \Leftrightarrow \frac {\dist{P_1 P_2}} {\dist{r}} - 1 \ge \sqrt{\frac 1 {p^{-1} - 1} \frac {\mass{M_2}} {\mass{M_1}}} \\
& \Leftrightarrow \frac {\dist{r}} {\dist{P_1 P_2}} \le \frac 1 {1 + \sqrt{\frac 1 {p^{-1} - 1} \frac {\mass{M_2}} {\mass{M_1}}}} \\
\end{align*}

$\force{F_1} = \mathcal G \frac {\mass{M_1} \mass{m}} {\dist{a}^2}$
$\force{F_2} = \mathcal G \frac {\mass{M_2} \mass{m}} {\dist{r}^2}$

\begin{align*}
\force{F_1} = \force{F_2}
& \Leftrightarrow \frac {\mass{M_1}} {\dist{a}^2} = \frac {\mass{M_2}} {\dist{r}^2} \\
& \Leftrightarrow \dist{r} = \dist{a} \sqrt{\frac {\mass{M_2}} {\mass{M_1}}} \\
\end{align*}



\section{Thrust}


\subsection{Set up}

We consider a rocket with center of mass $\posit{R}$. From $\delay{t}$ to
$\delay{t}+\delay{dt}$, it ejects $\mass{\dm}$ of its fuel with relative
velocity $\speed{v_e}$; its center of mass is noted $\posit{F}$. We will
use $\posit{G}$ to refer to the center of mass of the system containing
both the rocket and the ejected fuel.


\subsection{Derivation of exerted force}

At time $\delay{t}$, we have $\speed{v_G}(\delay{t}) =
\speed{v_R}(\delay{t}) = \speed{v_F}(\delay{t})$ so, at time
$\delay{t}+\delay{dt}$:
\begin{align*}
\speed{v_R}(\delay{t}+\delay{\dt}) \times \mass{m}
+
\speed{v_F}(\delay{t}+\delay{\dt}) \times \mass{\dm}
&=
\speed{v_G}(\delay{t}+\delay{\dt}) \times (\mass{m}+\mass{\dm})
\\
%
%
\speed{v_R}(\delay{t}+\delay{\dt}) \times (\mass{m}+\mass{\dm})
+
\speed{v_e} \times \mass{\dm}
&=
\speed{v_G}(\delay{t}+\delay{\dt}) \times (\mass{m}+\mass{\dm})
&& \text{with } \speed{v_F} = \speed{v_R} - \speed{v_e}
\\
%
%
\speed{v_R}(\delay{t}+\delay{\dt}) \times \mass{m}
+
\speed{v_e} \times \mass{\dm}
&=
\speed{v_G}(\delay{t}+\delay{\dt}) \times \mass{m}
&& \text{because } \mass{m}+\mass{\dm} \simeq \mass{m}
\\
%
%
\speed{v_R}(\delay{t}+\delay{\dt}) \times \mass{m}
+
\speed{v_e} \times \mass{\dm}
&=
\speed{v_R}(\delay{t}) \times \mass{m}
+
\force{F} \delay{\dt}
&&
\text{with } \speed{v_G}(\delay{t}+\delay{\dt}) = \speed{v_G}(\delay{t}) + \accel{\dot {v_G}} \delay{\dt}
\\
%
%
\frac {\speed{v_R}(\delay{t}+\delay{\dt}) - \speed{v_R}(\delay{t})} {\delay{\dt}}
&=
- \speed{v_e} \times \frac {\mass{\dm}} {\delay{\dt}} \times \frac 1 {\mass{m}} + \frac {\force{F}} {\mass{m}}
&& \text{by dividing by } \delay{\dt}
\\
%
%
\eqtag{eql:accel}
\mass{m} \accel{\dot {v_R}}
&=
\underbrace{- \speed{v_e} \dot m}_{\force{F_t}}
+ \force{F}
\end{align*}


\subsection{Specific impulse}

In KSP, the engines are defined by their maximum thrust $\force{F_t}$
and their $\delay{I_{\mathrm{sp}}}$ (“atmosphereCurve” in config
files). The SPecific Impulse is just the force exerted per unit (in
weight) of fuel used, i.e. $\delay{I_{\mathrm{sp}}} = \frac {\force{F_t}}
{\dot m \accel{g}}$. Thus:
\[
\begin{array}{lr}
\dot m
= \frac {\force{F_t}} {\delay{I_{\mathrm{sp}}} \times \accel{g}}
,
&
\speed{v_e}
= - \delay{I_{\mathrm{sp}}} \times \accel{g}
\end{array}
\]


\subsection{Tsiolkovsky rocket equation}

We can now integrate~(\ref{eql:accel}) to get the velocity change of a rocket:
\begin{align*}
\speed{\Delta v}
&= \int_{\delay{0}}^{\delay{t}} \accel{\dot{v_R}} \delay{\dt} \\
&= \int_{\delay{0}}^{\delay{t}} \left(
	- \speed{v_e} \frac {\dot m} {\mass{m}}
	+ \frac {\force{F}} {\mass{m}}
\right) \delay{\dt} \\
&= \left[- \speed{v_e} \ln(\mass{m})\right]_{\delay{0}}^{\delay{t}}
+ \underbrace{
	\int_{\delay{0}}^{\delay{t}} \frac {\force{F}} {\mass{m}} \delay{\dt}
}_{\speed{I}_{\force{F}}(\delay{t})} \\
\eqtag{eql:thrust}
&= \speed{v_e} \ln \frac {\mass{m}(\delay{0})} {\mass{m}(\delay{t})} + \speed{I}_{\force{F}}(\delay{t})
\end{align*}

This formula is usually applied when $\force{F}$ can be ignored (e.g. in
orbit). Note that, during ascent, $\frac {\force{F}} {\mass{m}} = - \frac
{\mathcal G \mass{M}} {(R + z)^2}$ leads to a quadratic differential
equation of second order which is difficult to solve.

Conversely, we can compute the amount of fuel to eject to reach a
given speed:
\[
\mass{\Delta m} = \mass{m} \left(1 - e^{-\frac {\speed{\Delta v}} {\speed{v_e}}}\right)
\]
