\chapter{Orbits}
\banner
\chaptquote{Johannes Kepler}{There is a force in the earth which causes the moon to move.}



\section{Orbital mechanics}

\begin{remark}
We study orbital mechanics before the ascent since we can always pretend
the space craft is in orbit. When it is resting on the launchpad, it is
at its apoapsis but stays there because of the ground. This apoapsis
is raised during the ascent and the gravity turn is simply raising
the periapsis.
\end{remark}


\subsection{Set up}

Consider two mass points $(\posit{P},\mass{M})$ and $(\posit{S},\mass{m})$
with $\mass{M} \gg \mass{m}$ and $\dist{r} = \dist{PS}$. We consider
the gravitational force exerted by $\posit{P}$ over $\posit{S}$. Using
Newton's Second Law, we have $\accel{\ddot {\vec r}} = - \frac {\mathcal
G \mass{M}} {\dist{r}^2} \hat r$.


\subsection{Orbital elements}

\begin{figure}[H]
	\centering
	\begin{tikzpicture}
		\pgfmathsetmacro{\a}{3}
		\pgfmathsetmacro{\e}{0.5}
		\pgfmathsetmacro{\pe}{\a*(1-\e)}
		\pgfmathsetmacro{\ap}{\a*(1+\e)}
		\pgfmathsetmacro{\f}{\e*\a}

		% orbit
		\node[boint=O]  (O) at (0,0) {};
		\node          (F1) at (-\f,0) {};
		\node[boint=F] (F2) at ( \f,0) {};
		\orbit[red]{F1}{\ap}{\pe}{0}{360};

		\orbitpoint[point=P]{F1}{\ap}{\pe}{45}{P}{}
		\node[roint=X] (X) at (\a,0) {};

		% true anomaly
		\draw (X) -- (F2) -- (P);
		\markangle{X}{F2}{P}{$\theta$}{0.5}

		% eccentric anomaly
		\draw (X) -- (O) -- (P);
		\markangle{X}{O}{P}{$E$}{0.5}
	\end{tikzpicture}
\end{figure}

\begin{itemize}
\item semimajor
\item semiminor
\item true anomaly
\item eccentric anomaly
\item inclination
\item argument of periapsis
\item longitude of ascending node
\item eccentricity
\item apoapsis
\item periapsis
\end{itemize}


\subsection{Kepler's second law}

We use polar coordinates centered in $\posit{P}$. According to Newton's
second law, $\accel{\ddot{\vec r}} = - \frac {\mathcal G \mass{M}}
{\dist{r}^2} \hat r$. Thus $2 \speed{\dot r} \dot \theta + \dist{r}
\ddot \theta = \accel{0}$. In other terms: $\frac 1 {\dist{r}} \frac {\d}
{\delay{\dt}} \Big(\dist{r}^2 \dot \theta\Big) = 0$. This means that the
area velocity $\dot {\mathcal A} = \frac 1 2 \dist{r}^2 \dot \theta$
is constant.


\subsection{\emph{Vis viva} equation}

The principle of energy conservation tells us that:
\[
E
=
\frac 1 2 \mass{m} \speed{v}^2
- \frac {\mathcal G \mass{M} \mass{m}} {\dist{r}}
\text{ is constant}
\eqtag{eql:orbenergy}
\]

We need two particular points to apply it. Let us consider the two points
where $\dist{r}$ is extremal, namely the apoapsis and periapsis. There,
$\speed{\dot r} = \speed{0}$ meaning that $\speed{\dot{\vec r}} = \dist{r}
\dot \theta \hat \theta$. We can then note that, at these two points:
\[
\dist{\vec r} \times \speed{\dot{\vec r}}
=
\dist{r}^2 \dot \theta \underbrace{\hat r \times \hat \theta}_1
= 2 \dot{\mathcal A}
\]

This means that $\dist{r_a} \speed{v_a} = \dist{r_p} \speed{v_p}$. Now,
we use the~(\ref{eql:orbenergy}) of energy and get:
\begin{align*}
\frac 1 2 \mass{m} \speed{v_a}^2
- \frac {\mathcal G \mass{M} \mass{m}} {\dist{r_a}}
&=
\frac 1 2 \mass{m} \speed{v_p}^2
- \frac {\mathcal G \mass{M} \mass{m}} {\dist{r_p}}
\\
\speed{v_a}^2 - \speed{v_r}^2
&=
2 \mathcal G \mass{M} \left(
	\frac 1 {\dist{r_p}}
	-
	\frac 1 {\dist{r_a}}
\right)
\\
\left(1 - \frac {\dist{r_a}^2} {\dist{r_p}^2}\right) \speed{v_a}^2
&=
2 \mathcal G \mass{M} \left(
	\frac 1 {\dist{r_p}}
	-
	\frac 1 {\dist{r_a}}
\right)
\\
\speed{v_a}^2
&=
2 \mathcal G \mass{M} \left(
	\frac 1 {\dist{r_p}}
	-
	\frac 1 {\dist{r_a}}
\right)
\frac {\dist{r_p}^2} {\dist{r_p}^2 - \dist{r_a}^2}
\\
&=
2 \mathcal G \mass{M}
\frac {\dist{r_a} - \dist{r_p}} {\dist{r_p} \dist{r_a}}
\frac {\dist{r_p}^2} {\dist{r_p}^2 - \dist{r_a}^2}
\\
&=
2 \mathcal G \mass{M}
\frac {1} {\dist{r_a}}
\frac {\dist{r_p}} {\dist{r_p} + \dist{r_a}}
\\
&=
\mathcal G \mass{M}
\frac {2 \dist{a} - \dist{r_a}} {\dist{r_a} \dist{a}}
\\
&=
\mathcal G \mass{M}
\left(
\frac 2 {\dist{r_a}}
-
\frac 1 {\dist{a}}
\right)
\end{align*}

We now have an expression of $\speed{v_a}$ that depends only on
$\dist{r_a}$ and $\dist{a}$. Let us use~(\ref{eql:orbenergy}) of energy
again on the apoapsis and an arbitrary point:
\begin{align*}
\frac 1 2 \mass{m} \speed{v_a}^2
- \frac {\mathcal G \mass{M} \mass{m}} {\dist{r_a}}
&=
\frac 1 2 \mass{m} \speed{v}^2
- \frac {\mathcal G \mass{M} \mass{m}} {\dist{r}}
\\
\speed{v}^2
&=
\mathcal G \mass{M} \left(
	\frac 2 {\dist{r}}
	-
	\frac 2 {\dist{r_a}}
\right)
+ \speed{v_a}^2
\\
&=
\mathcal G \mass{M} \left(
	\frac 2 {\dist{r}}
	-
	\frac 1 {\dist{a}}
\right)
\eqtag{eql:visviva}
\end{align*}



\section{Inclination}



\section{State prediction}



\section{Inferring an orbit}
